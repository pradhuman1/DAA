\documentclass[conference]{IEEEtran}
\IEEEoverridecommandlockouts
% The preceding line is only needed to identify funding in the first footnote. If that is unneeded, please comment it out.
\usepackage{cite}
\usepackage{amsmath,amssymb,amsfonts}
\usepackage{graphicx}
\usepackage{textcomp}
\usepackage{xcolor}
\usepackage{titlesec}

\usepackage{textcomp}
\usepackage{epsfig}
\usepackage{algpseudocode}
\usepackage{pgfplots}
\usepackage{tikz}
\pgfplotsset{width=10cm,compat=1.9}
 \usepgfplotslibrary{external}
\usepackage{amsmath}
\usepackage{mathtools}
\DeclarePairedDelimiter{\floor}{\lfloor}{\rfloor}
\usepackage[linesnumbered,ruled,vlined]{algorithm2e}
\def\BibTeX{{\rm B\kern-.05em{\sc i\kern-.025em b}\kern-.08em
    T\kern-.1667em\lower.7ex\hbox{E}\kern-.125emX}}

\usepackage[ruled,vlined]{algorithm2e}

\tikzexternalize 
\begin{document}

\title{ Check weather an integer repeated more than n/2 times in sorted array\\
\text{\Large{DAA ASSIGNMENT-4 , GROUP 1}}
}
\author{\IEEEauthorblockN{Harsh Mahajan}
\IEEEauthorblockA{ \text{IIB2019001}}
\and
\IEEEauthorblockN{Pradhuman Singh Baid}
\IEEEauthorblockA{ \text{IIB2019002}}
\and
\IEEEauthorblockN{Tauhid Alam}
\IEEEauthorblockA{ \text{BIM2015003}}
}

\maketitle

\begin{abstract}
This Paper contains the algorithm to find if a given integer x appears more than n/2 times in a sorted array of n integers. Divide and conquer approach is taken to solve it.
\end{abstract}

\section{Introduction}
Divide and conquer is the way of solving a problem by dividing the problem into smaller subproblems.

It is a technique in which the main problem is divided into smaller subproblems and then solving the smaller subproblems and combining them to find the solution of the problem.

In this paper, we have to find if a given integer x appears more than n/2 times in a sorted array of n integers


\section{Algorithmic Design}

In the problem we need to find weather a given integer x in appearing more than n/2 times in a sorted array or not.Efficiently this problem can be solved using binary search which is a divide and conquer algorithm.

First We divided the array in two parts and checked the middle element.
If this element is greater than or equal the integer x then we would focus on the left sub-array otherwise right sub-array.

Now, The size of focused sub-array is reduced to half of original array.

And then we will recursively apply this approach on the focused array until its size is reduced to 1. 

When the size of resultant array is '1' it means that we are on the index where the integer x appeared 1st time. 
Let name that index as 'left'.

Now,we need to check weather it appeared more than n/2 times or not.For that we would add n/2 to 'left' and name the result 'right'.As it is provided that array is sorted and if 'x' appeared more than n/2 times then it must also present at index 'right'.

And finally we will print the result according to its presence on index 'right'.


% Continue.....

\section{Pseudo Code}
\\

   Array ← sorted array of size n\\\
   target ← Target integer 'x' \\\
   
   solve function  \\
   \indent
   \hspace{0.5 cm}
    long long int mid ← middleEl  \\
    \indent \hspace{0.5 cm}
    long long int left ← 0  \\
    \indent \hspace{0.5 cm}
    long long int mid ← n  \\
    \indent \hspace {0.5cm}
    while left \textless right  \\
    \indent \hspace{1cm}
        mid = left + (left+right)/2  \\
        \indent \hspace{1cm}
        if (target \textless= Array[mid]) \\ 
        \indent \hspace{1.5 cm} right = mid \\ 
        \indent \hspace{1cm}
        else \\
        \indent \hspace{1.5 cm} left = mid + 1\\
        
    \indent \hspace {0.5cm} right = left + n/2 \\
    \indent \hspace {0.5cm} if(Array[right]=target) \\
    \indent \hspace{1cm} PRINT "YES" \\
    \indent \hspace {0.5cm} else \\
    \indent \hspace{1cm} PRINT NO \\  
    
main function \\
\indent \hspace{0.5 cm} call solve function \\ 
end procedure
% Continue.....

\section{Complexity Analysis}

\subsection{Time Complexity}\\
In this algorithm at each iteration the array is divided by 2. Suppose length of array is n and after K iteration the length of array becomes 1 and then we get\\
n/2^{k} = 1\\
n = 2^{k}\\
log(n) = log(2^{k})\\
log(n) = klog(4)\\
k=log(n)\\

Therefore, Time complexity is O(nlog(n))\\

\includegraphics[width=3.5in,height=2.5in]{logn.png}

\subsection{Space Complexity}\\
In this method, the space complexity would be O(n) as we have used an array of size n and some variables of constant space.  \\

Therefore, Space complexity is O(n)\\

\section{Conclusion}
Through this assignment we conclude that fastest way to find an element is using divide and conquer alogrithm.
\section{REFERENCES}
\textit{https://en.wikipedia.org/wiki/Divide-and-conquer_algorithm }\\
\textit{https://www.geeksforgeeks.org/divide-and-conquer-algorithm-i}\\
\end{document}